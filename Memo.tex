
\documentclass[a4paper,12pt]{article}


%\usepackage[spanish,activeacute]{babel} 
\usepackage[utf8]{inputenc} 
\usepackage{latexsym,amsfonts,amsmath,mathrsfs,amssymb}
\usepackage{verbatim}
\usepackage{fancyvrb}
\usepackage{graphicx}
\usepackage{makeidx}        
\usepackage{multicol}        



\author{Pedro Parrado Rodríguez\\* {\small PhD}}


\title{{\bf{Memory}}} 

% La fecha de entrega del informe y el número de la versión (1ª, 2ª, 3ª,..)
\date{{\scriptsize }}
\oddsidemargin0cm
\textwidth16cm


% Comenzamos el documento 
%jbg%\usepackage{Sweave}
\begin{document}

% Pone el título que hemos introducido arriba
\maketitle

\tableofcontents
\newpage


%\chapter{First Year}
\section{Warm up}
\subsection{Lookup tables}
\subsection{MonteCarlo simulations with Lookup tables}
\section{Surface Code}
\subsection{MLE Decoder}
\subsection{MonteCarlo simulation}
\subsection{Measurement errors Decoder}
\subsection{MonteCarlo simulation}

\section{Color Code}
\subsection{Rescaling decoder}
\subsubsection{Outline of the decoder}

\begin{enumerate}
\item Read the Syndrome from the errors in the qubits.
\item Assign an initial splitting (or initial split probabilities).
\item Update the splittings (or the split probabilities) until convergence.
\item Decode the cells independently according to the splitting from the previous step.
\item Create the new rescaled code, and assign to each logical qubit the probability of a logical error in the cell.
\item Apply the decoder again, until the size of the code is small enough to apply a complete lookuptable.
\item Apply the corrections of the higher levels to the lower ones.


\end{enumerate}

\subsubsection{Hard splitting method}
This algorithm for splitting is based on changing each individual splitting to a better one asuming the others are constant, until convergence. Therefore, the core of the algorithm is to compute the probability of a given splitting:

\begin{equation}
p(s_0^u|s_1s_2s_1's_2')=\frac{p(s_0^u|s_1s_2)p(s_0^l|s_1's_2')}{p(s_0^u|s_1s_2)p(s_0^l|s_1's_2')+[1-p(s_0^u|s_1s_2)][1-p(s_0^l|s_1's_2']}
\end{equation}


\begin{equation}
p(s_0^u|s_1s_2)=\frac{p(s_0^us_1s_2)}{p(s_0^u=1,s_1s_2)+p(s_0^u=0,s_1s_2)}
\end{equation}




% IMAGEN

%\begin{figure}[ht!]
%\begin{center}
%\includegraphics[height=90mm]{IMAGEN.png}
%\end{center}
%\end{figure}
%



%  TABLA

%\begin{table}[h]
%\begin{center}
%\begin{tabular}{|c|c|c|c|c|c|c|}
%\hline
%$		$&$		$&$		$&$		$&$		$&$		$&$		$\\
%\hline
%\end{tabular}
%\end{center}
%\end{table}





\end{document}